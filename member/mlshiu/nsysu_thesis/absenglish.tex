% Abstract in English
\renewcommand{\thesisabstracthead}{ABSTRACT}
\renewcommand{\thesistitle}{\thesistitleenglish}
\begin{thesisabstract}
\\
In this paper, we investigate rescaling of energy features for noise-robust speech recognition.
The performance of the speech recognition system will degrade very quickly by the influence of environmental noise.
As a result, speech robustness technique has become an important research issue for a long time.
However, many studies have pointed out that the impact of speech recognition under the noisy environment is enormous.
Therefore, we proposed the data-driven energy features rescaling (DEFR) to adjust the features.
The method is divided into three parts, that are voice activity detection (VAD), piecewise log rescaling function and
parameter searching algorithm.
The purpose is to reduce the difference of noisy and clean speech features.
We apply this method on Mel-frequency cepstral coefficients (MFCC) and Teager energy cepstral coefficients (TECC), and we compare the proposed method with mean subtraction (MS) and mean and variance normalization (MVN).
We use the Aurora 2.0 and Aurora 3.0 databases to evaluate the performance.
From the experimental results, we proved that the proposed method can effectively improve the recognition accuracy.

  \vspace{\baselineskip}
  \noindent
  \textbf{Keyword:} \engKeywords


\end{thesisabstract}

