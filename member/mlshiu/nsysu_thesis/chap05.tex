\chapter{結論與未來展望}
\label{ch:conclusion}
\section{結論}
自動語音辨識系統中,前端特徵擷取的方式對辨識率的影響相當大,
然而特徵向量中最重要的參數為對數能量與第零階倒頻譜參數。
這兩種參數為語音訊號在各個頻帶間的整體表現,因此極具重要性。
在本論文中,我們針對這兩種能量特徵提出了資料驅動能量特徵重刻法,
並以~MFCC~與~TECC~為基準對其能量特徵做調整。
此方法藉由~VAD~偵測出語音與非語音出現之段落,再利用分段對數尺度函數給予不同的權重,
重新計算其能量值,以補償語音在噪音環境下的失真。
而分段對數尺度函數的設計理念來自於雜訊語音其能量特徵值相對較高,非語音處之能量特徵值普遍偏低的特性,
希望增加語音與非語音之能量的差異性,故給予不同尺度的權重。
其函數所使用的參數是由參數搜尋法自動決定。
最後,我們採用~Aurora 2.0~與~Aurora 3.0~語料庫探討此方法在含噪音的環境下是否可以成功達到補償的效果,
並且與其他常用的特徵處理方法比較。
從實驗結果我們可以看出本論文所提出之方法不論是在~Aurora 2.0~或~Aurora 3.0~語料庫中,
皆有非常卓越的表現,這也證實了能量特徵的調整對於噪音強健性的影響非常大。

\section{未來展望}
資料驅動能量特徵重刻法的實驗結果雖然有大幅度的改善,但是仍然有進步的空間。
而其改善的方向可以~VAD~為主要目標,
主要原因為本論文所使用之~VAD~只考慮到低頻譜的能量強度,並未考慮到高頻的能量值。
因此大部分的能量皆會被誤判為語音的能量,如此一來,能量重刻所使用的下降尺度將會比較小,
而導致噪音的能量值下降幅度不夠,很難鑑別語音與非語音的能量值。
但是從另一個角度來看,假設語音的能量被誤判為非語音的能量比例較高,
將導致語音的能量降低過度,失去原有的特性。
所以一個準確的~VAD~對本論文的方法是相當重要的,
若能提高準確度,辨識效能也會相對的有所突破。