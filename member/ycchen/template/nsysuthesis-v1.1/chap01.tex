\chapter{Introduction}
\label{ch:intro}






\section{Motivation}
\label{sec:motivation}


隨著科技的進步,我們能夠接觸到的眼界也越來越廣泛。在現實生活之中,所能接觸到不單只是本國人,也可能會接觸外籍人士。在這樣的情況之下,如何能夠與外國人士溝通就是所要面臨的重要課題,如果能有專業的翻譯在其間居中協調,就能使得人們的溝通更加無國界化,可惜的是傳統人工翻譯,需花費較高的成本,且不符合對話者的私密性,而其花費更不是一般社會大眾所能負擔,因此利用機器翻譯系統能夠有效節省花費成本,更能夠勝任這樣的重要角色。


如今隨著網路的日益發達,可以察覺人們對於網路日趨依賴,越來越多的網路應用隨著時代潮流逐漸興起,商用機器翻譯廠商也樂意提供網路平台,以利於人們免費使用機器翻譯系統,但是顯而易見的是不同廠商的機器翻譯系統,都有著其各自擅長的領域以及句型,但這同時也意味著沒有一個線上機器翻譯系統是能夠將翻譯原始句上傳至翻譯網頁,就能確保得到信賴的結果。在這樣的情況之下,如何能夠利用不同線上翻譯系統將之整合成單一翻譯結果,以此方法可以得到一個值得信賴的翻譯結果,可以省去使用者大部份的比對時間,因此本研究所欲探討的問題即是線上翻譯系統的整合應用。

\section{Introduction of the System}
\label{sec:contributions}

目前有很多的研究都發現,多個機器翻譯系統的整合,相較於單一機器翻譯系統的結果為佳。因此本研究中,將提出一個機器翻譯整合系統,針對不同的線上翻譯系統進行整合。其大致流程如下:首先利用SRILM\cite{stolcke2002sel}所訓練之語言模型當作評分標準,決定系統所認定最佳之翻譯,更進一步參考了語音辨識中WER(Word Error Rate)的方法,將線上機器翻譯系統錯誤分成三大類,分別是Substitution、Insertion及Deletion,分別針對這三種不同的錯誤來進行處理。


在Substitution方面,本系統引進了其他翻譯句中可能是正確翻譯的字,但是卻沒有出現在最佳翻譯,考慮該字為系統誤譯成其他字因此尋找最佳翻譯中可能的錯誤翻譯,將之進行替換。


在Insertion方面,則是利用其他翻譯假說中緊連的字考慮其為片語的可能性,但是卻沒有同時出現在最佳翻譯中,以此考慮最佳翻譯中的字並將之延伸成片語可能性。


在Deletion方面,則是進一步將翻譯假說中多餘翻譯字,或者是錯誤翻譯字進行刪除的動作。


\section{Organization of the Thesis}
\label{sec:organization}

以下為本研究的編排方式:在第二章,講述關於本研究中一些相關之文獻探討,第三章部份講述本研究所用翻譯系統整合流程及方法,第四節部份講述針對其所做之評估結果,第五節則做為總結。

 